
\documentclass{article}
\usepackage{amsmath}
\usepackage{amssymb}
\usepackage{amsfonts}
\usepackage{physics}
\usepackage{graphicx}
\usepackage{hyperref}
\usepackage{xcolor}

\title{Improved Equation Rendering Test}
\author{Theory of Everything}
\date{\today}

\begin{document}

\maketitle

\section*{Testing Improved Equation Rendering}

This document tests the improved rendering of complex mathematical equations in LaTeX and PDF output.


\subsection*{Einstein Field Equations}

The Einstein field equations relate the geometry of spacetime to the distribution of matter within it.

\begin{align}
G_{\mu\nu} + \Lambda g_{\mu\nu} = \frac{8\pi G}{c^4} T_{\mu\nu}
\end{align}


\subsection*{Schrödinger Equation}

The Schrödinger equation describes how the quantum state of a physical system changes over time.

\begin{align}
i\hbar\frac{\partial}{\partial t}\Psi(\mathbf{r},t) = \hat{H}\Psi(\mathbf{r},t)
\end{align}


\subsection*{Maxwell's Equations}

Maxwell's equations describe how electric and magnetic fields are generated by charges, currents, and changes of each other.

\begin{align}
\begin{align} \nabla \cdot \mathbf{E} &= \frac{\rho}{\epsilon_0} \\ \nabla \cdot \mathbf{B} &= 0 \\ \nabla \times \mathbf{E} &= -\frac{\partial\mathbf{B}}{\partial t} \\ \nabla \times \mathbf{B} &= \mu_0\mathbf{J} + \mu_0\epsilon_0\frac{\partial\mathbf{E}}{\partial t} \end{align}
\end{align}


\subsection*{Dirac Equation}

The Dirac equation is a relativistic wave equation that describes the behavior of fermions.

\begin{align}
(i\gamma^\mu \partial_\mu - m)\psi = 0
\end{align}


\subsection*{Path Integral Formulation}

The path integral formulation of quantum mechanics describes the amplitude for a particle to travel from one point to another as a sum over all possible paths.

\begin{align}
\langle q_f, t_f | q_i, t_i \rangle = \int_{q(t_i)=q_i}^{q(t_f)=q_f} \mathcal{D}q(t) \exp\left(\frac{i}{\hbar}S[q]\right)
\end{align}

\end{document}